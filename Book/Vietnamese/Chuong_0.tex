\chapter{Hello world}
\newpage

\section{Về lập trình nhúng}
    
Đặc trưng của lập trình nhúng là viết chương trình để điều khiển phần cứng, ví dụ như chương trình điều khiển động cơ bước chẳng hạn. Với phần mềm ứng dụng như trên máy tính mà phần cứng yêu cầu giống nhau (màn hình, chuột, CPU...) và đòi hỏi nặng về khả năng tính toán của CPU. Còn với chương trình nhúng thì phần cứng của nó cực kì đa dạng, khác nhau với mỗi ứng dụng như chương trình điều khiển động cơ hoặc chương trình đọc cảm biến, nó không đòi hỏi CPU phải tính toán quá nhiều, chỉ cần quản lý tốt phần cứng bên dưới. 
    
Có 2 khái niệm là \textit{Firmware}, ý chỉ chương trình nhúng, và \textit{Software}, chương trình ứng dụng trên máy tính, được đưa ra để người lập trình dễ hình dung, nhưng không cần thiết phải phân biệt rõ ràng.
    
Khi lập trình hệ thống nhúng, việc biết rõ về phần cứng là điều cần thiết. Bởi bạn phải biết phần cứng của mình như thế nào thì bạn mới điều khiển hoặc quản lí tốt nó được. Tốt nhất là làm trong team hardware một thời gian rồi nhảy qua team firmware, hoặc làm song song cả hai bên nếu bạn có thể.

\section{Về ngôn ngữ C trong lập trình nhúng}
    
Ngôn ngữ C cho phép tương tác rất mạnh tới phần cứng, mạnh thế nào thì hồi sau sẽ rõ, thế nên nó thường được lựa chọn trong các dự án lâp trình nhúng. Ngoài ra có thể dùng C++ và Java nhưng mình ít xài chúng nên không đề cập ở đây.
    
Việc học C cơ bản mình sẽ không đề cập tới vì tài liệu nó nhiều lắm, các bạn có để xem và làm vài bài tập sử dụng được ngôn ngữ này. Lưu ý là ranh giới giữa việc \textbf{biết} và \textbf{sử dụng được} ngôn ngữ C là việc bạn có làm bài tập hay không. Về cú pháp thì nó quanh đi quẩn lại chỉ là khai báo biến, rồi mấy vòng lặp for, while hoặc rẽ nhánh if, else chẳng hạn, nhưng \textbf{kỹ năng} sử dụng C để giải quyết một vấn đề thì cần nhiều bài tập để trau dồi.
    
Tóm lại: \textit{bạn không thể đọc sách học bơi là biết bơi, không thể đọc kiếm phổ là thành cao thủ.}
\section{Về phần cứng để demo}

Trong phần này mình sẽ sử dụng 2 con Arduino Mega để demo cho nhanh. Bạn nên mua để dành nháp một vài cái project nào đấy. Mà thiệt tình thì nên hạn chế sử dụng Arduino vì nó làm bạn có thói quen sử dụng thư viện chùa với lại chẳng biết gì mấy về phần cứng, đến lúc Arduino chưa có thư viện thì không biết phải viết thế nào. Bạn có thể tìm hiểu các loại chip khác như STM32 (nó rất mạnh trong tầm giá của nó), PIC(nó bền và ổn định) hay xài kit Tiva cũng được. Ngoài ra phần thiên về lập trình không cần phần cứng sẽ sử dụng chương trình DevC++ (nhớ tạo project C, không cần C++), những cái cơ bản các bạn nên tự tìm hiểu, vì những cái đó tài liệu nó rất nhiều, và bạn cũng có thể tự mò cho quen.

\textit{Arduino là món mì ăn liền, sử dụng nó nhanh chóng nhưng nó cũng đầy điểm yếu.}
\section{Về Tiếng Anh, vâng tiếng Anh...}
    
Mấy ngành khác thì mình không rành chứ mà làm nhúng mà bạn không biết Tiếng Anh là tự níu chân mình lại. Vì mỗi linh kiện điện tử đều kèm theo một cái bảng thông tin đặc tính là datasheet, cái nào phức tạp thì sẽ kèm theo một cái hướng dẫn sử dụng là user manual. Và tất nhiên 96.69\% chúng được viết bằng tiếng Anh, còn lại là tiếng Trung Quốc. Và tin buồn là code trong lập trình nhúng đều biết bằng tiếng Anh, tin buồn hơn nữa là tài liệu, sách hướng dẫn, các diễn đàn sử dụng tiếng Anh rất nhiều và nhiều cái rất hay.
    
Tóm lại là không biết nó thì công việc của bạn bị cản trở rất nhiều, phụ thuộc rất nhiều vào google dịch củ chuối. Hãy dừng một bước để học tiếng anh và tiến 3 bước trong con đường sự nghiệp. Chí ít bạn phải đọc được datasheet mà không cần tra quá nhiều từ, đọc được cuốn sách như clean code chẳng hạn, hoặc viết email cho thằng bán linh kiện ở Trung Quốc vì kiểu gì sau này bạn cũng đặt hàng ở bên đó.
    
\textit{Nhớ rằng tiếng Anh là công cụ để sử dụng. Như người ta học đi xe máy để đi lại nhanh hơn. Hãy học tiếng Anh để làm việc ngon lành hơn.}


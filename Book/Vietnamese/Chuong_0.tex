\section{Mục tiêu của tài liệu}

Tài liệu này dành cho những bạn có định hướng theo đuổi con đường lập trình nhúng và cho các thiết bị IoT, không dành cho những bạn nào mới bắt đầu hoặc mới tìm hiểu. Ở đây mình sử dụng những kĩ năng, kinh nghiệm đã được áp dụng trong những dự án thực tế. Để đọc được tài liệu này bạn cần có kiến thức căn bản về C/C++, Arduino, điện tử.

\newpage

\section{Về lập trình nhúng}
    
Đặc trưng của lập trình nhúng là viết chương trình để điều khiển phần cứng, ví dụ như chương trình điều khiển động cơ bước chẳng hạn. Với phần mềm ứng dụng như trên máy tính mà phần cứng yêu cầu giống nhau (màn hình, chuột, CPU...) và đòi hỏi nặng về khả năng tính toán của CPU. Còn với chương trình nhúng thì phần cứng của nó cực kì đa dạng, khác nhau với mỗi ứng dụng như chương trình điều khiển động cơ hoặc chương trình đọc cảm biến, nó không đòi hỏi CPU phải tính toán quá nhiều, chỉ cần quản lý tốt phần cứng bên dưới. 
    
Có 2 khái niệm là \textit{Firmware}, ý chỉ chương trình nhúng, và \textit{Software}, chương trình ứng dụng trên máy tính, được đưa ra để người lập trình dễ hình dung, nhưng không cần thiết phải phân biệt rõ ràng.
    
Khi lập trình hệ thống nhúng, việc biết rõ về phần cứng là điều cần thiết. Bởi bạn phải biết phần cứng của mình như thế nào thì bạn mới điều khiển hoặc quản lí tốt nó được. Tốt nhất là làm trong team hardware một thời gian rồi chuyển sang team firmware, hoặc làm song song cả hai bên nếu bạn có thể.

\section{Về ngôn ngữ C trong lập trình nhúng}
    
Ngôn ngữ C cho phép tương tác rất mạnh tới phần cứng nên nó thường được lựa chọn trong các dự án lâp trình nhúng. Ngoài ra có thể dùng C++ và Java nhưng mình ít dùng chúng nên không đề cập ở đây.
    
Việc học C cơ bản mình sẽ không đề cập tới vì tài liệu đã rất nhiều, các bạn có để xem và làm vài bài tập sử dụng được ngôn ngữ này. Lưu ý là ranh giới giữa việc \textit{biết} và \textit{sử dụng được} ngôn ngữ C là việc bạn có làm bài tập hay không. Về cú pháp thì nó quanh đi quẩn lại chỉ là khai báo biến, rồi mấy vòng lặp for, while hoặc rẽ nhánh if, else chẳng hạn, nhưng \textit{kỹ năng} sử dụng C để giải quyết một vấn đề thì cần nhiều bài tập để trau dồi.
    
\section{Về phần cứng để demo}

Trong phần này mình sẽ sử dụng 1 con Arduino Uno để demo những đoạn code có liên quan. Bạn nên mua để dành nháp một vài cái project nào đấy. Ưu điểm của nó là hỗ trợ cho những bạn mới bước chân vào lập trình, những phần phức tạp đã được làm sẵn. Bạn có thể viết ngay một ứng dụng nào đó mà không cần phải tìm hiểu nhiều.

Và đó cũng chính là nhược điểm của Arduino. Nếu bạn ỷ lại nó mà không tìm hiểu sâu hơn thì khó lòng mà có thể tiến xa hơn được. Đặt trường hợp bạn phải làm việc với những dự án thực tế, có độ phức tạp cao, nếu dựa vào những thư viện Arduino cung cấp sẵn thì sẽ không thể hoàn thành được dự án.

Bạn có thể tìm hiểu các loại chip khác như STM32 (nó rất mạnh trong tầm giá của nó), PIC(nó bền và ổn định) hoặc các dòng chip của Texas Instrument. 

Khi có một đoạn code để demo thuần túy là C mình mình dùng phần mềm DevC++.

\section{Về Tiếng Anh}
    
Nếu bạn muốn theo đuổi lập trình chuyên nghiệp, kể cả lập trình nhúng hoặc các ngành công nghệ thông tin khác thì phải trau dồi tiếng Anh. Vì đặc điểm của ngành này có thể liệt kê ra như sau:

\begin{itemize}
    \item Bắt nguồn từ các nước Âu-Mỹ.
    \item Thay đổi nhanh và liên tục.
    \item Tài liệu, sách vở, cộng đồng, dự án có sẵn... đa số đều viết bằng tiếng Anh.
\end{itemize}

Để theo đuổi ngành thì bạn cần học những kiến thức mới để đáp ứng với thời cuộc, cần trao đổi với các đồng nghiệp khác để giải quyết những vấn đề, đôi khi là tìm cách gỡ lỗi trong chương trình.

Đa số các lỗi bạn gặp khi lập trình đều đã có người mắc phải và có cách giải quyết ở một nơi nào đó, việc của bạn nếu bị kẹt là tìm trên Internet cách gỡ lỗi đó, và thường thì phải biết tiếng Anh. Việc mắc ở một lỗi nào đấy và không gỡ được làm bạn dễ chán nản.


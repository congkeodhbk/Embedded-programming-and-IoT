\chapter{Debug}

Debug là việc các bạn sửa lỗi code vừa mới viết xong để cho nó chạy đúng với mục đích của bạn. Ở đây mình sẽ giới thiệu một công cụ để các bạn debug được dễ dàng hơn. 

Những chip mạnh như dòng STM32 có hỗ trợ để các bạn có thể chạy từng dòng lệnh và kiểm tra giá trị của từng biến, từng thanh ghi, nhưng với Arduino thì không hỗ trợ việc này.

Thư viện mình hỗ trợ sau đây sẽ tương tự với hàm printf và scanf trong C, giúp bạn giao tiếp với MCU thông qua câu lệnh.

Video demo mình sẽ để ở đường link sau: 

Dưới đây là phần mô tả để các bạn có thể hiểu rõ hơn về thư viện này.

\section{Con trỏ hàm}

Ngoài con trỏ tới một biến thì C còn một loại con trỏ nữa là con trỏ hàm.

Trong thư viện Debug này, chương trình có thể bắt được lệnh mà bạn gửi xuống từ máy tính, sau đó bạn phải tự viết một đoạn code khác để xử lí lệnh này. Con trỏ hàm sẽ giúp bạn cho thư viện Debug biết cần phải làm gì sau khi bắt được lệnh bằng cách trỏ tới hàm này.

Các hàm xử lí do bạn viết này thường được gọi là hàm \textit{callback}. Chúng xuất hiện rất nhiều trong lập trình nhúng, khi bạn sử dụng thư viện của người khác và muốn tùy biến lại theo yêu cầu của ứng dụng.\newline

\begin{lstlisting}
void console_handler(char *result){

}

void setup() {
    debug_init(&Serial, 9600, console_handler);
}
\end{lstlisting}

Ở trên khai báo hàm \textit{console\_handler} và hàm này được truyền như một tham số cho hàm \textit{deubg\_init}.

\section{Xử lí câu lệnh}

__check_cmd("CMD "): Kiểm tra xem câu lệnh bắt được có phải là lệnh "CMD " không. Tất nhiên sẽ có nhiều lệnh khác nhau được truyền xuống, bạn có thể thay "CMD " bằng lệnh nào tùy thích.

sscanf và __param_pos: hai lệnh này để bắt được tham số các bạn truyền theo sau lệnh "CMD " ("CMD " là một lệnh bất kỳ). Ví dụ như bạn muốn chớp tắt led với tần số 10 Hz hay 100 Hz, hoặc cho động cơ quay 1 vòng hay 10 vòng... Câu lệnh truyền xuống thì giống nhau nhưng tham số khác nhau dẫn đến MCU chạy khác một cách khác nhau.


\chapter{Viết thư viện}

Nếu bạn đi theo con đường lập trình chuyên nghiệp thì bạn phải chuẩn bị cho mình một bộ thư viện để khi cần thì có thể lôi ra sử dụng. Trên Arduino được hỗ trợ rất nhiều nhưng không có nghĩ là đúng và  đủ. Tất cả thư viện đó đều có thể có lỗi và đôi khi có những thư viện chưa được hỗ trợ. Hoặc trường hợp thường gặp như thư viện của cảm biến DHT11 đã có trên Arduino nhưng dự án của bạn lại sử dụng chip khác thì cần biết làm sao để chuyển từ thư viện Arduino sang loại chip mà bạn đang sử dụng.

Trên Arduino thư viện được viết bằng C++ nhưng mà mình sẽ hướng dẫn bạn viết bằng C, vì nó dễ hơn. Sau này rành rồi thì bạn cũng nên tự học C++ vì nó có nhiều chiêu hay hơn C (nó hơn hai đấu cộng lận đó).
\section{Cấu trúc thư viện}

Một thư viện có thể có rất nhiều file trong đó. Nhưng cơ bản thì nó có 2 file chính: file code \textit{.c} và file header \textit{.h}. File code thì tất nhiên là để viết code rồi. Nhưng mà có ai sử dụng thư viện mà lôi code ra đọc từ đầu đến cuối rồi mới sử dụng đâu. Đó là lí do sinh ra file \textit{.h}. Nó là nơi \textbf{khai báo hàm}, bao gồm các hàm mà trong thư viện mà bạn có thể xài, các cấu trúc dữ liệu... và đặc biệt là phải có các đoạn hướng dẫn sử dụng, để người dùng chỉ cần đọc hướng dẫn thôi là sử dụng được cái thư viện đó rồi. Cho nên thư viện của bạn viết, ngoài đáp ứng 3 tiêu chí chạy được, dễ thay đổi, dễ hiểu ra, còn thêm một yêu cầu cho thư viện là dễ sử dụng nữa.

Đối với Arduino thì bạn cần đặt 2 file này vào cùng thư mục của với file .ino, còn như sử dụng các chip khác thì mỗi thư viện cần có một thư mục riêng.

Một ví dụ về thư viện:
\begin{lstlisting}
//file example.h
#ifndef _EXAMPLE_H
	#define _EXAMPLE_H

	//Initiate example library 
	void example_init(void);

	//Example function
	void example_function(void);

#endif
\end{lstlisting}

\begin{lstlisting}
//file example.c
#include "example.h"

void example_init(void){
	//do something
}

void example_function(void){
	//do something
}

\end{lstlisting}


Lưu ý: đoạn macro
\begin{lstlisting}
//file example.h
#ifndef _EXAMPLE_H
	#define _EXAMPLE_H
	.
	.
	.
#endif
\end{lstlisting}
Là rất quan trọng, mỗi file \textit{.h} nên có một đoạn này, và \_EXAMPLE\_H thay đổi tùy theo tên hàm của bạn.

Có thể giải thích như sau: trong C nó có một bộ tiền xử lý (preprocessor), là trước khi biên dịch code của bạn thì nó sẽ làm việc của cái bộ tiền xử lý này trước. Cặp khai báo \#ifndef \#endif sẽ đi chung với nhau. Đoạn khai báo trên có nghĩa là: Nếu chưa định nghĩa \_EXAMPLE\_H thì định nghĩa \_EXAMPLE\_H và mấy cái bên dưới, cho tới chỗ \#endif thì dừng.

Giả sử mình có file a.h , b.h được khai báo như sau:

\begin{lstlisting}
//file a.h
int A=0;
\end{lstlisting}
\begin{lstlisting}
//file b.h
#include "a.h"
int B=0;
\end{lstlisting}
và có thêm một file c.h
\begin{lstlisting}
//file c.h
#include "a.h"
#include "b.h"
int C=0;
\end{lstlisting}
bạn có thể thấy, khi c \textit{\#include "a.h"} thì nó đã khai báo một biến A trong đó rồi. Khi c \textit{\#include "b.h"} thì nó khai báo tiếp 2 biến A, B nữa, vậy biến A được khai báo 2 lần và sẽ gây lỗi.

Tất nhiên là bạn không nên khai báo biến trong file \textit{.h}, nhưng ở đây là ví dụ về trường hợp sẽ sinh ra lỗi. Nếu sử dụng cặp khai báo  \#ifndef \#endif thì sẽ giải quyết được chuyện này. 
\begin{lstlisting}
//file a.h
#ifndef _A_H
	#define _A_H
	int A=0;
#endif
\end{lstlisting}
\begin{lstlisting}
//file b.h
#ifndef _B_H
	#define _B_H
	#include "a.h"
	int B=0;
#endif
\end{lstlisting}

khi c \textit{\#include "a.h"} thì nó sẽ khai báo biến A và định nghĩa chuỗi \_A\_H (định nghĩa này là ghi nhận nó đã từng xuất hiện trên đời). Khi c \textit{\#include "b.h"}, thì đầu tiên b.h sẽ  \textit{\#include "a.h"}, nhưng nó thấy \_A\_H đã từng xuất hiện rồi nên không khai báo phần bên dưới nữa. Rồi cuối cùng nó chỉ định khai báo biến B thôi.

Tóm lại là đoạn macro đó giúp bạn không bị khai báo lại khi mà \textit{\#include} chồng chéo nhiều thư viện khác nhau.

Một lưu ý khác là thư viện C của bạn có thể được sử dụng bởi một chương trình C++, nên cách viết để C++ có thể đọc nó như sau:
\begin{lstlisting}
//file example.h
#ifndef _EXAMPLE_H
	#define _EXAMPLE_H
	
	#ifdef __cplusplus
 		extern "C" {
	#endif

	//Initiate example library 
	void example_init(void);

	//Example function
	void example_function(void);
	
	#ifdef __cplusplus
	}
	#endif
	
#endif
\end{lstlisting}
hãy làm quen với bộ tiền xử lý vì nó được sử dụng nhiều lắm.
\section{Viết thư viện theo phong cách OOP}

Rồi, sau khi bạn có thói quen chia chương trình thành cách file thư viện nhỏ rồi, không còn nhét mọi thư vôi main.c nữa thì bạn có thể đến bước này.

Lập trình hướng đối tượng (OOP) là một bước tiến của lập trình cấu trúc ( ngôn ngữ C là hướng cấu trúc, C++ là hướng đối tượng). OOP có nhiều ưu điểm hơn, sử dụng được cho những chương trình lớn hơn. C không phải là ngôn ngữ hướng đối tượng nhưng bạn có thể bắt chước phong cách OOP và cuộc sống sẽ dễ dàng hơn.

Bạn hoàn toàn có thể sử dụng C++ cho Arduino nhưng các chip khác thường vẫn sử dụng C, và bạn làm việc với C một thời gian mà chuyển lên C++ mới thấy nó rất phê, giống như chuyển từ assemply qua C vậy. Còn mà xài C++ ngay từ đầu luôn thì nó cũng thường thôi.

Vào vấn đề chính. Mình muốn viến thư viện điều khiển một cái relay (rờ le). Bình thường thì nó tắt. Khi nào bạn muốn bật nó lên thì ngoài gọi lệnh như turn\_relay\_on() ra bạn còn phải truyền thời gian khi nào nó tắt nữa, không để nó chạy hoài, vì nhiều khi relay đang kéo tải công suất lớn như máy bơm hay đèn gì đấy. Và tất nhiên là có thể dừng nó bất kì lúc nào và điều khiển một lúc nhiều relay khác nhau.

Trước tiên là khai báo một struct cho mỗi relay:
\begin{lstlisting}
//relay.h

//low active or high active
#define ON HIGH
#define OFF LOW
typedef struct{
	uint8_t pin;
	uint8_t state;
	uint32_t on_state_begin;
	uint32_t on_state_timeout;
}relay_t;
\end{lstlisting}
\textit{pin} là xem relay đó nối với chân nào của Arduino, state là trạng thái nó đang ON hay OFF, \textit{on\_state\_begin} là thời điểm bắt đầu ON, còn \textit{on\_state\_timeout} là thời gian ON của nó, hết cái đó nó sẽ tắt.

Sau đó sẽ khai báo các hàm sử dụng struct này:
\begin{lstlisting}
//relay.h
//initiate relay first
void relay_init(relay_t *relay, uint8_t pin);
void turn_relay_on(relay_t *relay, uint32_t timeout);
void turn_relay_off(relay_t *relay);
//return true if relay high state timeout
bool is_relay_on_state_timeout(relay_t *relay);
\end{lstlisting}

Và rồi trong chương trình chính mình sẽ sử dụng nó như sau.
\begin{lstlisting}

#include "relay.h"

relay_t relay;

void setup(){
	relay_init(&relay, 5); //pin 5 control this relay
	delay(1000);
	turn_relay_on(&relay, 1000);
}

void loop(){
	if(is_relay_on_state_timeout(&relay)){
		turn_relay_off(&relay);
	}
}

\end{lstlisting}

Chương trình chính ở setup() sẽ khởi động biến relay và bật relay đó lên và trong loop() liên tục hỏi coi nó có timeout chưa, nếu có thì tắt relay đó đi. Lưu ý là hàm is\_relay\_on\_state\_timeout() chỉ trả về true nếu relay đó đang ON và đã hết thời gian, trả về false nếu đang OFF hoặc chưa timeout.

Bạn hoàn toàn có thể khai báo một mảng relay\_t relay[8] để điều khiển một lúc 8 cái relay. Mỗi cái cần một lần khởi tạo và gắn chân riêng, con trỏ truyền vào có thể là \&relay[0] để điều khiển relay số 0.

Cái tinh thần của OOP trong đây là bạn khai báo một biến (coi nó như một đối tượng), rồi sau đó không động chạm gì đến thành phần bên trong của biến đó mà chỉ sử dụng các hàm được khai báo sẵn để tương tác với biến đó. Các thành phần bên trong biến sẽ được truy cập tại thư viện. 

Viết thư viện theo kiểu này sẽ đảm bảo bao đóng của dữ liệu để đối tượng nó được an toàn. Như bạn thấy biến state trong kiểu relay\_t là một biến nội bộ lưu lại trạng thái của relay. Như hàm turn\_relay\_on() sẽ có lệnh digitalWrite(pin, ON) và gán biến state=ON bên trong. Nếu bạn tự tiện gán relay.state=OFF trong chương trình chính thì sẽ bị lỗi (xem code bên dưới).

Việc bao đóng dữ liệu sẽ giúp các module độc lập hơn. Bạn không cần quan tâm trong biến relay\_t có cái gì ở trỏng. Chỉ cần khai báo, rồi gọi hàm truyền nó vào. Như vậy nó đảm bảo được nguyên tắc A biết B làm được cái gì chứ không biết B làm điều đó như thế nào.


Sau đây là phần code cho relay đó:
\begin{lstlisting}
//relay.c
void relay_init(relay_t *relay, uint8_t pin)
{
	relay->pin=pin;
	pinMode(relay->pin, OUTPUT);
	digitalWrite(relay->pin, OFF);
	relay->state=OFF
}

void turn_relay_on(relay_t *relay, uint32_t timeout)
{
	digitalWrite(relay->pin, ON);
	relay->state=ON;
	relay->on_state_begin=millis();
	relay->on_state_timeout=timeout;
}

void turn_relay_off(relay_t *relay)
{
	digitalWrite(relay->pin, OFF);
	relay->state=OFF;
}

bool is_relay_on_state_timeout(relay_t *relay)
{
	if(relay->state==ON) 
	{
		if((millis() - relay->on_state_begin)\
						 > relay->on_state_timeout)
			return true; //return and exit
	}
	return false;
}

\end{lstlisting}

Tất nhiên là mình viết sơ sơ để các bạn dễ đọc thôi, còn nhiều cái khác để làm cho thư viện này chạy ổn. Như kiểu người ta chưa init nó lên mà đã gọi hàm ON OFF thì phải làm sao? Haha, tới đây bạn sẽ phát hiện là trong relay.c sửa tá lả làm bên trong chương trình chính không sửa gì vẫn chạy được. Bạn có thể không cần phải điều khiển relay thật luôn mà nháp trước với một con led cũng được.

Hãy nghiên cứu phần này để cuộc sống nó dễ chịu hơn.

\documentclass[12pt,a5paper]{book}
\usepackage[utf8]{vietnam}
\usepackage{import}

\usepackage[utf8]{vietnam}
\usepackage[utf8]{inputenc}
\usepackage{amsmath}
\usepackage{verbatim}
\usepackage{amsfonts}
\usepackage{tikz}
\usetikzlibrary{arrows,automata,shapes.geometric}
\usepackage{amssymb}
\usepackage{graphicx}
\usepackage{listings}
\usepackage{textcomp}
\usepackage[hidelinks]{hyperref}

\renewcommand{\familydefault}{\sfdefault}
\usepackage[left=2cm,right=1.5cm,top=1.5cm,bottom=1.5cm]{geometry}

 \renewcommand\lstlistingname{Quelltext} % Change language of section name

\lstset{ % General setup for the package
	language=C,
	basicstyle=\small\sffamily,
	numbers=left,
 	numberstyle=\tiny,
	frame=tb,
	tabsize=4,
	columns=fixed,
	showstringspaces=false,
	showtabs=false,
	keepspaces,
	%commentstyle=\color{green},
	%keywordstyle=\color{blue}
}
\tikzstyle{startstop} = [rectangle, rounded corners, minimum width=2cm, minimum height=0.25,text centered, draw=black]
\tikzstyle{io} = [trapezium, trapezium left angle=70, trapezium right angle=110, minimum width=3cm, minimum height=1cm, text centered, draw=black]
\tikzstyle{process} = [rectangle, minimum width=2cm, minimum height=0.25, text centered, draw=black]
\tikzstyle{decision} = [diamond, minimum width=2cm, minimum height=0.25, text centered, draw=black]
\tikzstyle{arrow} = [thick,->,>=stealth]
\pagenumbering{gobble}
\pagenumbering{arabic}

\title{Lập Trình Nhúng và IoT}
\date{\today}
\author{Nguyễn Thành Công}
\setcounter{secnumdepth}{0}

\begin{document}
\pagenumbering{gobble}
\maketitle
 \import{Vietnamese}{Chuong_0}
\tableofcontents
 \import{Vietnamese}{Chuong_1}

 \import{Vietnamese}{Chuong_2}

 \import{Vietnamese}{Chuong_3}

% \import{Vietnamese}{Chuong_5}

\begin{thebibliography}{9}
\bibitem{Clean Code} 
Robert C. Martin. 
\textbf{Clean Code: A Handbook of Agile Software Craftsmanship}. 
 
\bibitem{Code Complete} 
Steve McConnell.
\textbf{Code Complete: A Practical Handbook of Software Construction, Second Edition}.

\bibitem{lập trình nhúng} 
Hoàng Trang, Bùi Quốc Bảo.
\textbf{Lập trình hệ thống nhúng}.

\bibitem{nhập môn lập trình}
Trần Đan Thư, Nguyễn Thanh Phương, Đinh Bá Tiến, Trần Minh Triết.
\textbf{Nhập môn lập trình}.

\bibitem{Kĩ Thuật Lập trình}
Trần Đan Thư, Nguyễn Thanh Phương, Đinh Bá Tiến, Trần Minh Triết, Đặng Bình Phương.
\textbf{Kĩ thuật lập trình}.

\bibitem{Computer Networking: A Top-Down Approach}
James Kurose, Keith Ross.
\textbf{Computer Networking: A Top-Down Approach}.

\end{thebibliography}

\end{document}
